% !TeX program = lualatex
% !BIB program = biber
% Lualatex is important to render TTF fonts; with pdflatex it's just the regular one
% ratio 16:9 -- https://tex.stackexchange.com/questions/14336/

% compile two versions, inspired by https://tex.stackexchange.com/a/1501
% use the script "compile-pdf.sh"
\newif\ifhandout
% if flags.tex does not exist, create an empty file to be able to compile in TeXstudio
\input{flags}

\ifhandout
\documentclass[12pt,aspectratio=169,handout]{beamer}
\else
\documentclass[12pt,aspectratio=169]{beamer}
\fi



% TODO change "leftfootertext" to your liking
\newcommand{\leftfootertext}{\insertsubtitle}  % just the \title{} text by default
%\newcommand{\leftfootertext}{RNNs and encoder-decoder architectures}  % Your name, for instance


% ------- RUB specifics ----------
% adjust for 16:9
% https://tex.stackexchange.com/questions/354022/modifying-the-margins-of-all-slides-in-beamer
\setbeamersize{text margin left=0.3cm,text margin right=4.5cm} 


% use Metropolis as the basis theme
\usetheme[subsectionpage=progressbar]{metropolis}
% blocks with background globally
\metroset{block=fill}


\usepackage{fontspec}
% RUB fonts need to be installed
% 'UprightFont = * Light' makes sure that the base font is RubFlama Light, which looks
% lighter than RubFlama Regular (would be too thick for slides)
\setsansfont[Scale=MatchLowercase, UprightFont = * Light, BoldFont = * Bold]{RubFlama}
%\setsansfont{Arial} % Open source alternative if you don't have RubFlama

% RUB color scheme
% Dark blue: 0; 53; 96; #003560
\definecolor{RUBDarkBlue}{RGB}{0, 53, 96}

% Light yellow (table fill, etc.); 238; 250; 196; #EEFAC4
\definecolor{RUBLightYellow}{RGB}{238, 250, 196}

%Light green: 141; 174; 16
\definecolor{RUBLightGreen}{RGB}{141, 174, 16}


\setbeamercolor{titlelike}{fg=RUBDarkBlue}
\setbeamercolor{subtitle}{fg=RUBLightGreen}
\setbeamercolor{separation line}{fg=RUBLightGreen}
\setbeamercolor{frametitle}{bg=white, fg=RUBDarkBlue}

% horizontal line on title page and sections
\setbeamercolor{alerted text}{fg=RUBLightGreen}


% Adjust footer bottom (too large by default)
\setbeamertemplate{footline}{%
  \begin{beamercolorbox}[wd=\textwidth, sep=2ex]{footline}%
    \usebeamerfont{page number in head/foot}%
    \usebeamertemplate*{frame footer}
    \hfill%
    \usebeamertemplate*{frame numbering}
  \end{beamercolorbox}%
}


% Lab name, numbering, etc. in footer
\setbeamertemplate{frame numbering}{TrustHLT --- Prof.\ Dr.\ Ivan Habernal \hspace*{1ex} \includegraphics[width=7em]{img/rub-logo.pdf}\hspace*{1ex}}

\setbeamertemplate{frame footer}{\hspace*{1ex}\insertframenumber \hspace*{2ex} \leftfootertext}

% adjust the background to be completely white
\setbeamercolor{background canvas}{bg=white}

% logos on the title page
\titlegraphic{%
	\begin{picture}(0,0)
		\put(435,0){\makebox(0,0)[rt]{\includegraphics[width=7em]{img/rub-logo.pdf}}}
		\put(435,-170){\makebox(0,0)[rt]{\includegraphics[width=4em]{img/logo-trusthlt.pdf}}}
		\put(435,-196){\makebox(0,0)[rt]{\includegraphics[width=9em]{img/logo-rctrust.pdf}}}
	\end{picture}%
}


% show TOC at every section start
\AtBeginSection{
	\frame{
		\vspace{2em}
		\sectionpage
		\hspace*{2.2em}\begin{minipage}{10cm}
			\tableofcontents[currentsection]
		\end{minipage}
	}
}

% TOC without subsection
\setcounter{tocdepth}{1} % only-- part,chapters,sections 

% bullet points: rectangles
\useinnertheme{rectangles}
\setbeamercolor{itemize item}{fg=RUBLightGreen}
\setbeamercolor{itemize subitem}{fg=RUBLightGreen}
% enumerate: blue background for better readability
\setbeamercolor{item projected}{bg=RUBDarkBlue}

% make boxes (example, block, etc.) background lighter for readability
\setbeamercolor{block title}{%
	use=normal text,
	fg=normal text.fg,
	bg=normal text.bg!90!fg % lighter background in block title
}
\setbeamercolor{block body}{
	use={block title, normal text},
	bg=block title.bg!30!normal text.bg % lighter background in block body
}


% RUB colors in blocks
\setbeamercolor{block title alerted}{%
	use={block title, alerted text},
	bg=RUBDarkBlue,
	%fg=RUBLightYellow % looks bad
	fg=white % better contrast
}

\setbeamercolor{block title example}{%
	use={block title, example text},
	fg=RUBLightGreen
}


% ------- end of RUB specifics ----------

% all itemize with pause by default
%\beamerdefaultoverlayspecification{<+->}


% typeset mathematics on serif
\usefonttheme[onlymath]{serif}

% better bibliography using biber as backend
\usepackage[natbib=true,backend=biber,style=authoryear-icomp,maxbibnames=30,maxcitenames=9,uniquelist=false,giveninits=true,doi=false,url=false,dashed=false,isbn=false]{biblatex}
% shared bibliography
\addbibresource{../../nlpwdl-bibliography.bib}
% disable "ibid" for repeated citations
\boolfalse{citetracker}



\usepackage{xspace}


% for derivatives, https://tex.stackexchange.com/a/412442
\usepackage{physics}

\usepackage{tikz}
\usetikzlibrary{matrix, positioning}
\usetikzlibrary{angles,quotes} % for angles
\usetikzlibrary{backgrounds} % background
\usetikzlibrary{decorations.pathreplacing} % curly braces
\usetikzlibrary{calligraphy}
\usetikzlibrary{calc} % for neural nets

% for plotting functions
\usepackage{pgfplots}
\usepgfplotslibrary{dateplot}

% sub-figures
\usepackage{caption}
\usepackage{subcaption}

% book tabs
\usepackage{booktabs}


% argmin, argmax
\usepackage{amsmath}
\DeclareMathOperator*{\argmax}{arg\!\max}
\DeclareMathOperator*{\argmin}{arg\!\min}
% softmax
\DeclareMathOperator*{\softmax}{soft\!\max}
% Mask
\DeclareMathOperator*{\mask}{mask}

% bold math
\usepackage{bm}

% for \mathclap
\usepackage{mathtools}

% algorithms
\usepackage[noend]{algpseudocode}


% for neurons and layers in tikz
\tikzset{
	neuron/.style={draw, rectangle, inner sep=2pt, minimum width=0.75cm, fill=blue!20},
	param/.style={draw, rectangle, inner sep=2pt, minimum width=0.75cm, fill=green!20},
	constant/.style={draw, rectangle, inner sep=2pt, minimum width=0.75cm, fill=black!15},
	% for citation nodes right top
	ref/.style={anchor = north east, text width=7.8cm, yshift=-1.3cm, xshift=-0.2cm, scale=0.5},
	state/.style={rectangle, inner sep=2pt, minimum width=0.75cm, fill=black!5},
}

% added in lecture 10
\tikzset{
	mtx/.style={
		matrix of math nodes,
		left delimiter={[}, right delimiter={]}
	},
	hlt/.style={opacity=0.1, line width=4 mm, line cap=round},
	hltr/.style={opacity=0.5, rounded corners=2pt, inner sep=-1pt}
}

% for strike-through text (added in Lecture 06)
\usepackage[normalem]{ulem}

% added in Lecture 7
% RNN
\DeclareMathOperator*{\rnn}{RNN}
% RNN star
\DeclareMathOperator*{\rnnstar}{RNN^{*}}
% bi-RNN
\DeclareMathOperator*{\birnn}{biRNN}


% added in Lecture 9
\usetikzlibrary{fit} % for hightligting by calling "fit"

% algorithms
\usepackage[noend]{algpseudocode}



\title{Natural Language Processing with Deep Learning}
\subtitle{Lecture 11 --- Retrieval augmented generation}
\date{January 15, 2026}
\author{Prof.\ Dr.\ Ivan Habernal}
\institute{
\texttt{www.trusthlt.org} \\
Trustworthy Human Language Technologies Group (TrustHLT) \\
Ruhr University Bochum \& Research Center Trustworthy Data Science and Security}



\begin{document}

\maketitle

\begin{frame}{Motivation\footnote{This lecture is heavily based on the excellent book by \citet{Jurafsky.Martin.2026}}}

Important function of large language models is to fill human information needs

For example factoid questions in short texts like the following:

\texttt{Where is the Louvre Museum located?}

To get an LLM to answer these questions, we can just prompt it!

\begin{tikzpicture}[overlay, remember picture]
\node at (current page.north east)[ref] {
\fullcite{Jurafsky.Martin.2026} \par};
\end{tikzpicture}

\end{frame}


\begin{frame}{Prompting LLM answers factoid questions}

For example a pretrained LLM that has been instruction-tuned on answering questions could directly answer the question

\texttt{Where is the Louvre Museum located?}

by performing conditional generation given this prefix, and take the response as the
answer.

This works because \pause
\begin{itemize}
\item LLMs have processed a lot of facts in their pretraining data, including the location of the Louvre, and have encoded this information in their parameters
\end{itemize}


\begin{tikzpicture}[overlay, remember picture]
\node at (current page.north east)[ref] {
\fullcite{Meng.et.al.2022.NeurIPS} \par};
\end{tikzpicture}

\end{frame}


\begin{frame}{Prompting LLM answers factoid questions}

This works because
\begin{itemize}
\item Factual knowledge of this type seems to be stored in the connections in the very large feedforward layers of transformers
\end{itemize}

But: knowledge stored in the feedforward weights of the LLM leads to a number of problems with prompting as a method for correctly generating factual texts or answers


\begin{tikzpicture}[overlay, remember picture]
\node at (current page.north east)[ref] {
\fullcite{Meng.et.al.2022.NeurIPS} \par};
\end{tikzpicture}

\end{frame}

\begin{frame}{Problem \#1 of simple prompting: LLMs hallucinate}
Hallucination is a response that is not faithful to the facts of the world


LMs sometimes give incorrect factual responses even when the correct facts are stored in the parameters

This seems to be caused by the feedforward layers failing to recall the knowledge stored in their parameters \citep{Jiang.et.al.2024.NAACL.hallucination}

\begin{tikzpicture}[overlay, remember picture]
\node at (current page.north east)[ref] {
\fullcite{Jiang.et.al.2024.NAACL.hallucination} \par};
\end{tikzpicture}


\end{frame}


\begin{frame}{Problem \#2 of simple prompting: Proprietary data}

Answer from pre-trained parameters does not allow us to ask questions about proprietary data

\pause

\begin{itemize}
\item proprietary data like personal email
\item healthcare application: LLM to medical records
\item company may have internal documents that contain answers for customer service or internal use
\item legal firms need to ask questions about legal discovery from proprietary documents
\end{itemize}

\pause

None of this data (hopefully) was in the large web-based corpora that large language models are pre-trained on
\end{frame}



\begin{frame}{Problem \#3 of simple prompting: Static knowledge}

LLMs were pretrained once, at a particular time

LLMs \textbf{cannot} talk about about rapidly changing information (like something that
happened last week) since they won't have up-to-date information from after their
release data

\end{frame}


\begin{frame}{One solution to these problems: RAG}

Give a language model external sources of knowledge, use those documents in answering questions

This method is called retrieval-augmented generation (RAG)

\begin{enumerate}
\item we use information retrieval (IR) techniques to retrieve documents that are likely to have information that might help answer the question
\item then we use LLM to generate an answer given these documents
\end{enumerate}

\end{frame}


\section{Information retrieval}


\begin{frame}{Information retrieval}

\begin{itemize}
\item User has an information need
\item And has some collection of documents
\item User wants to find a relevant document which satisfies their needs
\end{itemize}

Typical examples: Web search, RAG

\end{frame}


\begin{frame}{In most cases we do ranked retrieval}
\includegraphics[width=\linewidth]{img/ir1.jpg}

The retriever returns top-$k$ documents

These are ranked

We can show the user these, or some subset
\end{frame}





\begin{frame}{Document Relevance Score}
Goal is to assign a score to each document
for whether it meets the user's information
need

Instead, we just approximate this by the
textual similarity between the query and the
document
\end{frame}


\begin{frame}{Two architectures}

Sparse retrieval
\begin{itemize}
\item represent query and doc as vectors of word counts
\item weighted by tf-idf, BM25
\end{itemize}

Dense retrieval
\begin{itemize}
\item Use LLM to represent query and doc as embeddings
In both cases, similarity is dot product or cosine
between query and document representations
\end{itemize}


\end{frame}


\subsection{Sparse retrieval: the vector model of IR}


\begin{frame}{The vector space model of IR}
Represent a document as a vector of counts of the words it contains: Bag-of-words representation

\includegraphics[width=0.8\linewidth]{img/bow1.jpg}

\end{frame}


\begin{frame}{Term-document matrix}

\includegraphics[width=\linewidth]{img/matrix1.jpg}

\end{frame}




\begin{frame}{The vector space model of IR}
Represent a document as a vector of counts of the words it contains: Bag-of-words representation

\includegraphics[width=0.8\linewidth]{img/bow1.jpg}

\end{frame}


\begin{frame}{Term-document matrix}

\includegraphics[width=\linewidth]{img/matrix1.jpg}

\end{frame}



\begin{frame}{Visualizing document vectors}

In two dimensions only: \texttt{battle} and \texttt{fool}

\includegraphics[width=\linewidth]{img/bow2.jpg}

\end{frame}



\begin{frame}{Term weighting: tf-idf and BM25}

In IR, we don't use raw word counts like $(1, 114, 36, 20)$ for \emph{As You Like It}

Instead we compute a \textbf{term weight} for each document word

Two common term weighting schemes
\begin{itemize}
\item tf-idf
\item variant of tf-idf called BM25
\end{itemize}

term frequency \textbf{tf} and the inverse document frequency \textbf{idf}

\end{frame}


\begin{frame}{Term frequency}

The \textbf{tf} term tells us how frequent the word is

\pause

Words that occur more often in a document are likely to be informative about the document's contents

\pause

We usually use the $\log_{10}$ of the word frequency, rather than the raw count.

\begin{itemize}
\item The intuition is that a word appearing 100 times in a document doesn't make that word 100 times more likely to be relevant to the meaning of the document
\end{itemize}

\end{frame}

\begin{frame}{Term frequency}

So if we define $\mathrm{count}(t,d)$ as the raw count of term $t$ in document $d$, then $\mathrm{tf}_{t,d}$ is
\pause

$$
\mathrm{tf}_{t,d} =
\begin{cases}
1 + \log_{10} \mathrm{count}(t,d) & \text{if } \mathrm{count}(t,d) > 0\\
0 & \text{otherwise}
\end{cases}
$$
\end{frame}

\begin{frame}{Document frequency}
Document frequency $\mathrm{df}_t$ of a term $t$ is the number of documents the term $t$ occurs in
\pause

\begin{itemize}
\item Terms that occur in only a few documents are useful for discriminating those documents from the rest of the collection \pause
\item Terms that occur across the entire collection are not as helpful
\end{itemize}

\end{frame}


\begin{frame}{Inverse document frequency}
Document frequency $\mathrm{df}_t$ of a term $t$ is the number of documents the term $t$ occurs in


\begin{block}{Inverse document frequency of term $t$}
$$
\mathrm{idf}_t = \log_{10} \frac{N}{\mathrm{df}_t}
$$
where $N$ is the total number of documents
\end{block}

\pause

\begin{itemize}
\item The fewer documents in which a term occurs, the higher this weight
\item The lowest weight of 0 is assigned to terms that occur in every document
\end{itemize}

\end{frame}

\begin{frame}{tf-idf}

The tf-idf value for word $t$ in document $d$ is then the product
$$
\textrm{tf-idf}(t, d) = \mathrm{tf}_{t, d} \cdot \mathrm{idf}_t
$$
\end{frame}


\begin{frame}{Document scoring}

We have each document $d$ and the query $q$ represented as vectors $\bm{d} \in \mathbb{R}^n$ and $\bm{q} \in \mathbb{R}^n$

How do we measure their similarity score?

\pause

Cosine similarity
$$
\cos(\bm{q}, \bm{d}) = \frac{\bm{q} \cdot \bm{d}}{\norm{\bm{q}} \norm{\bm{d}}}
$$


\end{frame}

\section{Evaluation IR}


\begin{frame}{Evaluation}

The same \textbf{precision} and \textbf{recall} metrics we have been using

\pause
We make the assumption that each document returned by the IR system is \textbf{either relevant} to our purposes or \textbf{not relevant}

\pause
Unfortunately, these metrics do not adequately measure the performance of a system
that ranks the documents it returns. For comparing two ranked retrieval systems, we need a metric that prefers the one that ranks the relevant documents higher.

\end{frame}

\begin{frame}{Towards Mean Average Precision (MAP)}

\includegraphics[width=\linewidth]{img/map1.jpg}

\pause

Let $R_r$ be the set of relevant documents at or above $r$

\pause

Let $\mathrm{Precision}_r(d)$ be precision measured at the rank at which document $d$ was found

\end{frame}

\begin{frame}{Mean Average Precision (MAP)}

Let $R_r$ be the set of relevant documents at or above $r$

Let $\mathrm{Precision}_r(d)$ be precision measured at the rank at which document $d$ was found

\pause
The average precision $\mathrm{AP}$ for a single query is
$$
\mathrm{AP} = \frac{1}{\abs{R_r}} \sum_{d \in R_r} \mathrm{Precision}_r(d)
$$
\pause
and the Mean Average Precision for a set of queries $Q$ is
$$
\mathrm{MAP} = \frac{1}{\abs{Q}} \sum_{q \in Q} \mathrm{AP}(q)
$$
\end{frame}

\subsection{Dense information retrieval}

\begin{frame}{Vector representation}

Transformer-based model for representing both the query and the document collection

Recall SentenceBERT, but gazillions of other models...

\end{frame}

\begin{frame}{IR benchmark/leader-board example}


\includegraphics[width=\linewidth]{img/leaderbord.jpg}



\begin{tikzpicture}[overlay, remember picture]
\node at (current page.north east)[ref] {
Octen-Embedding-8B: ``Octen-Embedding-8B is a text embedding model designed for semantic search and retrieval tasks. This model is fine-tuned from Qwen/Qwen3-Embedding-8B and supports multiple languages, providing high-quality embeddings for various applications.'' --- ``Qwen3 Embedding model series is the latest proprietary model of the Qwen family, specifically designed for text embedding and ranking tasks'' \par};
\end{tikzpicture}



\end{frame}



\section{Retrieval Augmented Generation}

\begin{frame}{RAG in a nutshell}

\includegraphics[width=\linewidth]{img/rag1.jpg}

\end{frame}


\begin{frame}{RAG algorithm}

The idea of RAG is to condition on the retrieved passages, jointly with some prompt text, for example like
\texttt{Based on these texts, answer this question: ...}

Given a document collection $D$ and a user query $q$, the most basic RAG algorithm is
\begin{enumerate}
\item Call a retriever to return $R(q) = d_1, \ldots, d_k$, the top-$k$ relevant passages from $D$
\item Create a prompt that includes $q$ and the retrieved passages
\item Call an LLM with the prompt
\end{enumerate}


\end{frame}

\begin{frame}{LLM prompting with RAG}

\begin{block}{Schema of a prompt}
\noindent{}\texttt{retrieved passage 1 \\
retrieved passage 2 \\
... \\
retrieved passage k \\
Based on these texts, answer this question: What year
was the premiere of The Magic Flute?
}
\end{block}

\end{frame}

\begin{frame}{Extensions of basic RAG}

\pause

In \textbf{agent-based RAG}, the system decides when to call a \textbf{retrieval agent} and for which collection

\pause

There may be noise in the retrieved passages; some of
them may be irrelevant or wrong, or in an unhelpful order. How can we encourage
the LLM to focus on the good passages? Some RAG architectures add a \textbf{reranker}
that reranks or reorders passages after they are retrieved

\end{frame}

\begin{frame}{Knowledge citation (or Answer Attribution)}

\includegraphics[width=0.75\linewidth]{img/citation1.jpg}

\begin{tikzpicture}[overlay, remember picture]
\node at (current page.north east)[ref] {
\fullcite{gao-etal-2023-enabling} \par};
\end{tikzpicture}

\end{frame}



\begin{frame}{Knowledge citation (or Answer Attribution)}

The simplest way for generating knowledge citations is to specify it as part of the prompt. 

\begin{example}
\texttt{Write an answer for the given question using only the
provided search results (some of which might be irrelevant)
and cite them properly... Always cite for any factual claim}
\end{example}

\begin{tikzpicture}[overlay, remember picture]
\node at (current page.north east)[ref] {
\fullcite{gao-etal-2023-enabling} \par};
\end{tikzpicture}


\end{frame}

\begin{frame}{Answer Attribution can be Unfaithful}

``The aforementioned approaches do not account for attributions' faithfulness, i.e. whether the selected documents influence the LLM during the generation.''

``Indeed, the presence of an entailment relation or high semantic similarity does not imply that the retrieved document had an influence on the answer generation process.''


\begin{tikzpicture}[overlay, remember picture]
\node at (current page.north east)[ref] {
\fullcite{qi-etal-2024-model} \par};
\end{tikzpicture}

	
\end{frame}




\begin{frame}{First RAG papers}

``We demonstrated a \textbf{simple technique} to greatly improve factual unsupervised cloze QA by
providing context documents as additional inputs. We used oracle documents to establish an
upper bound to this improvement, and found that \textbf{using off-the-shelf information retrieval}
is sufficient to achieve performance on par with the supervised DrQA system. We also
investigated how brittle language models' factual predictions were to \textbf{noisy and irrelevant
context} documents.''


\begin{tikzpicture}[overlay, remember picture]
\node at (current page.north east)[ref] {
\fullcite{petroni2020how} \par};
\end{tikzpicture}

\end{frame}







\begin{frame}{License and credits}

	\begin{columns}
		\begin{column}{0.7\textwidth}
			Licensed under Creative Commons Attribution-ShareAlike 4.0 International (CC BY-SA 4.0)
		\end{column}
		\begin{column}{0.2\textwidth}
			\includegraphics[width=0.9\linewidth]{img/cc-by-sa-icon.pdf}
		\end{column}
	\end{columns}
	
	\bigskip
	
	Credits
	
	\begin{scriptsize}
		
		Ivan Habernal
		
		Content from ACL Anthology papers licensed under CC-BY \url{https://www.aclweb.org/anthology}

This lecture re-used materials from \citet{Jurafsky.Martin.2026}. They explicitly state: \emph{``Feel free to use the draft chapters and slides in your classes, print it out, whatever, the resulting feedback we get from you makes the book better!''}, see \url{https://web.stanford.edu/~jurafsky/slp3/}
		
	
	\end{scriptsize}
	
\end{frame}


\end{document}

