
% TODO change "leftfootertext" to your liking
\newcommand{\leftfootertext}{\insertsubtitle}  % just the \title{} text by default
%\newcommand{\leftfootertext}{RNNs and encoder-decoder architectures}  % Your name, for instance


% ------- RUB specifics ----------
% adjust for 16:9
% https://tex.stackexchange.com/questions/354022/modifying-the-margins-of-all-slides-in-beamer
\setbeamersize{text margin left=0.3cm,text margin right=4.5cm} 


% use Metropolis as the basis theme
\usetheme[subsectionpage=progressbar]{metropolis}
% blocks with background globally
\metroset{block=fill}


\usepackage{fontspec}
% RUB fonts need to be installed
% 'UprightFont = * Light' makes sure that the base font is RubFlama Light, which looks
% lighter than RubFlama Regular (would be too thick for slides)
\setsansfont[Scale=MatchLowercase, UprightFont = * Light, BoldFont = * Bold]{RubFlama}
%\setsansfont{Arial} % Open source alternative if you don't have RubFlama

% RUB color scheme
% Dark blue: 0; 53; 96; #003560
\definecolor{RUBDarkBlue}{RGB}{0, 53, 96}

% Light yellow (table fill, etc.); 238; 250; 196; #EEFAC4
\definecolor{RUBLightYellow}{RGB}{238, 250, 196}

%Light green: 141; 174; 16
\definecolor{RUBLightGreen}{RGB}{141, 174, 16}


\setbeamercolor{titlelike}{fg=RUBDarkBlue}
\setbeamercolor{subtitle}{fg=RUBLightGreen}
\setbeamercolor{separation line}{fg=RUBLightGreen}
\setbeamercolor{frametitle}{bg=white, fg=RUBDarkBlue}

% horizontal line on title page and sections
\setbeamercolor{alerted text}{fg=RUBLightGreen}


% Adjust footer bottom (too large by default)
\setbeamertemplate{footline}{%
  \begin{beamercolorbox}[wd=\textwidth, sep=2ex]{footline}%
    \usebeamerfont{page number in head/foot}%
    \usebeamertemplate*{frame footer}
    \hfill%
    \usebeamertemplate*{frame numbering}
  \end{beamercolorbox}%
}


% Lab name, numbering, etc. in footer
\setbeamertemplate{frame numbering}{TrustHLT --- Prof.\ Dr.\ Ivan Habernal \hspace*{1ex} \includegraphics[width=7em]{img/rub-logo.pdf}\hspace*{1ex}}

\setbeamertemplate{frame footer}{\hspace*{1ex}\insertframenumber \hspace*{2ex} \leftfootertext}

% adjust the background to be completely white
\setbeamercolor{background canvas}{bg=white}

% logos on the title page
\titlegraphic{%
	\begin{picture}(0,0)
		\put(435,0){\makebox(0,0)[rt]{\includegraphics[width=7em]{img/rub-logo.pdf}}}
		\put(435,-170){\makebox(0,0)[rt]{\includegraphics[width=4em]{img/logo-trusthlt.pdf}}}
		\put(435,-196){\makebox(0,0)[rt]{\includegraphics[width=9em]{img/logo-rctrust.pdf}}}
	\end{picture}%
}


% show TOC at every section start
\AtBeginSection{
	\frame{
		\vspace{2em}
		\sectionpage
		\hspace*{2.2em}\begin{minipage}{10cm}
			\tableofcontents[currentsection]
		\end{minipage}
	}
}

% TOC without subsection
\setcounter{tocdepth}{1} % only-- part,chapters,sections 

% bullet points: rectangles
\useinnertheme{rectangles}
\setbeamercolor{itemize item}{fg=RUBLightGreen}
\setbeamercolor{itemize subitem}{fg=RUBLightGreen}
% enumerate: blue background for better readability
\setbeamercolor{item projected}{bg=RUBDarkBlue}

% make boxes (example, block, etc.) background lighter for readability
\setbeamercolor{block title}{%
	use=normal text,
	fg=normal text.fg,
	bg=normal text.bg!90!fg % lighter background in block title
}
\setbeamercolor{block body}{
	use={block title, normal text},
	bg=block title.bg!30!normal text.bg % lighter background in block body
}


% RUB colors in blocks
\setbeamercolor{block title alerted}{%
	use={block title, alerted text},
	bg=RUBDarkBlue,
	%fg=RUBLightYellow % looks bad
	fg=white % better contrast
}

\setbeamercolor{block title example}{%
	use={block title, example text},
	fg=RUBLightGreen
}


% ------- end of RUB specifics ----------

% all itemize with pause by default
%\beamerdefaultoverlayspecification{<+->}


% typeset mathematics on serif
\usefonttheme[onlymath]{serif}

% better bibliography using biber as backend
\usepackage[natbib=true,backend=biber,style=authoryear-icomp,maxbibnames=30,maxcitenames=9,uniquelist=false,giveninits=true,doi=false,url=false,dashed=false,isbn=false]{biblatex}
% shared bibliography
\addbibresource{../../nlpwdl-bibliography.bib}
% disable "ibid" for repeated citations
\boolfalse{citetracker}



\usepackage{xspace}


% for derivatives, https://tex.stackexchange.com/a/412442
\usepackage{physics}

\usepackage{tikz}
\usetikzlibrary{matrix, positioning}
\usetikzlibrary{angles,quotes} % for angles
\usetikzlibrary{backgrounds} % background
\usetikzlibrary{decorations.pathreplacing} % curly braces
\usetikzlibrary{calligraphy}
\usetikzlibrary{calc} % for neural nets

% for plotting functions
\usepackage{pgfplots}
\usepgfplotslibrary{dateplot}

% sub-figures
\usepackage{caption}
\usepackage{subcaption}

% book tabs
\usepackage{booktabs}


% argmin, argmax
\usepackage{amsmath}
\DeclareMathOperator*{\argmax}{arg\!\max}
\DeclareMathOperator*{\argmin}{arg\!\min}
% softmax
\DeclareMathOperator*{\softmax}{soft\!\max}
% Mask
\DeclareMathOperator*{\mask}{mask}

% bold math
\usepackage{bm}

% for \mathclap
\usepackage{mathtools}

% algorithms
\usepackage[noend]{algpseudocode}


% for neurons and layers in tikz
\tikzset{
	neuron/.style={draw, rectangle, inner sep=2pt, minimum width=0.75cm, fill=blue!20},
	param/.style={draw, rectangle, inner sep=2pt, minimum width=0.75cm, fill=green!20},
	constant/.style={draw, rectangle, inner sep=2pt, minimum width=0.75cm, fill=black!15},
	% for citation nodes right top
	ref/.style={anchor = north east, text width=7.8cm, yshift=-1.3cm, xshift=-0.2cm, scale=0.5},
	state/.style={rectangle, inner sep=2pt, minimum width=0.75cm, fill=black!5},
}

% added in lecture 10
\tikzset{
	mtx/.style={
		matrix of math nodes,
		left delimiter={[}, right delimiter={]}
	},
	hlt/.style={opacity=0.1, line width=4 mm, line cap=round},
	hltr/.style={opacity=0.5, rounded corners=2pt, inner sep=-1pt}
}

% for strike-through text (added in Lecture 06)
\usepackage[normalem]{ulem}

% added in Lecture 7
% RNN
\DeclareMathOperator*{\rnn}{RNN}
% RNN star
\DeclareMathOperator*{\rnnstar}{RNN^{*}}
% bi-RNN
\DeclareMathOperator*{\birnn}{biRNN}


% added in Lecture 9
\usetikzlibrary{fit} % for hightligting by calling "fit"

% algorithms
\usepackage[noend]{algpseudocode}